\documentclass[12pt]{article}
\usepackage[margin=1in]{geometry}       
\usepackage{amsmath}               
\usepackage{fancyhdr}               
\usepackage{graphicx}               
\usepackage{cancel} 
\usepackage[utf8]{inputenc}
\usepackage[brazilian]{babel}
\usepackage{indentfirst}
\pagestyle{fancy} 
\fancyhead[LO,L]{João Roberto da S. P.}
\fancyhead[CO,C]{MATA48 - Arquitetura de Computadores}
\fancyhead[RO,R]{\today}
\fancyfoot[LO,L]{}
\fancyfoot[CO,C]{\thepage}
\fancyfoot[RO,R]{}
\renewcommand{\headrulewidth}{0.4pt}
\renewcommand{\footrulewidth}{0.4pt}

\begin{document}       
\title{Resumo 1}
\date{23 de junho de 2023}
\author{João Roberto da Silva Porto}

\begin{abstract}
Impactos socioambientais do ciclo de vida do hardware: análise e soluções 
\end{abstract}

\section*{A produção e o descarte do lixo eletrônico}
O paradigma central da quarta geração de computadores: os microprocessadores, tem seu surgimento intimamente conectado a evolução e barateamento dos de micro eletrônicos baseados em semicondutores.
Entretanto esse desdobramento histórico do segmento da informática, por muito tempo desprezou as consequências ambientais associados ao fim do uso desses bens. 

Desta forma, somente na última década legislação e regulamentação efetiva e em larga escala tem sido, ainda que de forma insipiente, imposta sobre os fabricantes de produtos eletrônicos. Uma problemática antiga,
que ainda não havia projetado todo seu potencial destrutivo. Ainda assim, essencial compreendendo desde a dificuldade do descarte em todo seu processo, desde a triagem ao desmonte e reaproveitamneto.
 Contudo, essas necessárias atitudes colidem com interesses corporativos.

 Uma vez que a alta minuciosidade e escala microscópica de muitos componentes ainda que valiosos elevam significantemente o trabalho associado a extreção de suas partes, e logicamente encarecem o processo.
 Outro choque de interesses, é verificado na lógica da obsolecencia planejada e do estimulo ao consumo ciclico de produtos, uma tendência facilmente observada no segmento de smartphones, onde o reparo, por exemplo é notoriamente dificultado por muitas empresas a fim de estimular a substituição do produto ao invês da
 alternativa ecologica. Assim, instigando hardware com ciclo de vida progressivamente mais curto. 

 Nos últimos anos, a cobrança estatal por maior responsabilidade ambiental tem forçado empresas a adotarem práticas que favoreçam um descarte apropriado, que favoreça a harmonia ambiental. Uma solução especialmente essencial, dado a natureza dificilmente reversivel dos danos causados pelo descarte impróprio de muitos componentes populares de eletrônicos, cujos os danos 
manifestam-se desde a poluição de lençois freáticos por substâncias cancerígenas até disrupção de habitats.

 Tendo em mente a alta complexidade do processo de triagem e reciclagem faz-se necessária a implantação da infraestrutura necessária para a reciclagem devida, 
 que, apesar de cara, mantém-se central para preservação do planeta, notavelmente com o crescimento exponencial verificado na produção desse tipo de dejeto nas últimas décadas.

\section*{Ponto de coleta mais próximo}
Casas Bahia Salvador - Itapuã, na Avenida Dorival Caymmi, 264.

\end{document}